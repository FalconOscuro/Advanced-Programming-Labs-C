\documentclass[Lab-C.tex]{subfiles}

\begin{document}
    \section{Copy}
        \subsection*{Question:}
        Complete the functionality inside of the \mintinline{cpp}{Copy(char filenamein[], char filenameout[])} function. 
        You need to add code that will try to open a text file given by the \mintinline{cpp}{filenamein} array as the name of the input file. 
        You need to add code to create and output file using the \mintinline{cpp}{filenameout} array as the name of the ouput file. 
        Then you can add code that will take each \textit{char} from the input file and put it in the output file.
            
        \subsection*{Solution:}
            \inputminted{cpp}{../02-Copy/Copy.cpp}%CPP file path here

        \subsection*{Test Data:}
            The program was tested with several scenarios:
            \begin{itemize}
                \item \textbf{Basic Input}:\\
                    The program was tested with the provided \textit{input.txt} file
                    all text was successfully copied to the newly created \textit{output.txt} file.
                
                \item \textbf{No Input File}:\\
                    If the \textit{input.txt} cannot be found, execution terminates.
                    
                \item \textbf{Blank File}:\\
                    Copying succeeds even if the file is empty.
            \end{itemize}
        
        \subsection*{Sample Output:}
            All data is output to the \textit{output.txt} file.

        \subsection*{Reflection:}
            There is another scenario where the file is too large to copy, this is handled
            if not enough memory can be allocated, but there is still unexpected behaviour if there are
            more characters than countable by a 64 bit unsigned integer, but since that is over
            18 exabytes, larger than the estimated size of the internet, 
            I don't think this will become an issue anytime soon.
\end{document}